\documentclass{article}
\usepackage[russian]{babel}
\usepackage{ragged2e}

\begin{document}
\begin{flushleft}

\section{}

Заведем ДО. В каждом узле будем хранить массив $dp[3][3]$, где $dp[i][j]$ -- количество способов добраться из клетки $L + i$ в клетку $R - j$ ($dp[i][j] = 0$ если $L + i$ или $R - j$ не принадлежат $[L, R]$ или содержат препятствия). В листах все значения кроме $dp[0][0]$ равны $0$. $dp[0][0] = 1$, если лист отвечает за клетку без препятствия. В остальных узлах мы можем посчитать значения по формуле: $t.dp[i][j] = l.dp[i][0] \cdot (r.dp[0][j] + r.dp[1][j] + r.dp[2][j]) + l.dp[i][1] \cdot (r.dp[0][j] + r.dp[1][j]) + l.dp[i][2] \cdot r.dp[0][j]$. При добавлении/удаление препятствия обновим значения на пути от соответсвующего листа до корня.

\section{}



\section{}

Преобразуем исходный массив так, чтобы его элементы принимали значения $[0, n)$ и если $x < y$ в исходном массиве, то $x' < y'$. Заведем массив $p$ в котором для каждого числа сохраним индекс его предыдущего вхождения ($-1$ если вхождение первое). Тогда для отрезка $[L, R]$ количество уникальных элементов это количество $x \in p: x < L$. Заведем ДО, узел которого хранит количество $x: L \le x \le R$. C помощью него мы можем за $O(\log n)$ узнать количество добавленных элементов $< x$. Будем последовательно добавлять в него элементы массива $p$. В момент, когда мы добавили $k$ элементов массива мы можем ответить на запрос для префикса массива длины $k$. Количество уникальных элементов на отрезке $[L, R]$ это количество элементов $< L$ на префиксе $R$ минус количество таких элементов на префиксе $L - 1$. Запомним все необходимые запросы и запомним ответы на них в процессе добавления элементов в ДО. Тогда после прохождения массива $p$ мы будем знать ответы на все запросы. 

\section{}

Для прибавления $x$ на пути $v\Rightarrow u$ прибавим $x$ на пути $root\Rightarrow v$ и $root\Rightarrow u$ и $-x$ на пути $root\Rightarrow lca(u, v)$. Для прибавления на пути от корня до вершины используем Link-Cut Tree.


\end{flushleft}
\end{document}